%
% File ACL2016.tex
%

\documentclass[11pt]{article}
\usepackage{acl2016}
\usepackage{times}
\usepackage{latexsym}
\usepackage{url}
\usepackage{booktabs}
\usepackage{graphicx}
\usepackage{color}
\usepackage{amsmath}
\aclfinalcopy 

\usepackage[authoryear]{natbib}
\usepackage{url}

\title{Influence of timing on Twitter engagement}
\author{Floris Bokx 5321107\\}
\date{January 16th 2024}

\begin{document}
\maketitle

%%% YOUR PART HERE
\begin{abstract}
In this research, we try to discover the optimal time of day to post on Twitter (Recently rebranded to X, but in this paper we will reference to the platform as Twitter). First, we look for related literature from earlier conducted studies about what the optimal time is to post on Twitter. We gather data from many different Tweets with their timestamp and retweet count. We gather this data using the Twitter API, which allows us to make requests about Twitter's data. Based on this, we create a results table with 24 timeslots (1 for each hour) and see which timeslot has the highest retweet count on average. We anticipate seeing more engagement towards the end of the afternoon at around 5pm-6pm. Our predicted results show that the optimal time is between 4pm and 5pm. The lowest retweeting activity can be found between 3am and 4am, with a significant difference between the two. We can draw conclusions from these results. For example, we can say that people are most active on Twitter right after work finishes. 

\end{abstract}

%% IMPORTANT: KEEP ALL SECTIONS (headers)
%% remove the 'red' text parts

\section{Introduction}
We have decided to research the influence of the time of day Tweets are posted, on their engagement. With Twitter (X) having over 350 million users around the world, it is a crucial platform for communication, and understanding when people engage with tweets is becoming essential. By posting content when your audience is most active, more people will engage with it ~\citep{AIContentfy:2023}. If more people engage, the tweet will also be pushed more by the algorithm reaching a wider audience.

\begin{itemize}
\item The research question is as follows: 

How does the timing of tweets impact the retweet engagement on Twitter, and what is the optimal time of day to post a tweet aiming to get the highest engagement possible? At what time do tweets receive the least amount of engagement?

\item Hypothesis: We think that there will be significant differences in Tweeting engagement between the different time slots. We think that tweets between 5pm and 6pm will have the highest average retweets, since around this time, a lot of users coming back from work are active on social media. We think that tweets between 3am and 4am will have the lowest engagement. This period of the day is called "dead of night". This is the time when fewest people are awake, so it will also have the lowest traffic on Twitter.


\end{itemize}
\section{Related Work}
Multiple past studies have already researched as to what the optimal timing for posting on Twitter is, showing different perspectives. ~\cite{Anoob:2023} states that the best time for engagement are early in the morning (7-9 am) and late at night (8-11 pm). This would suggest that users are more active before work and at the end of the day before going to sleep. "Does that mean these time slots are optimal for every time zone?", is a question that arises, since this is not talked about in the article.

On the other hand, ~\cite{Myers:2024} proposes a different time frame. She does specify time zones, and says 4 pm to 5 pm in Central European Time (CET) will reach the widest audience. Perhaps this difference could show a cultural difference compared to the rest of the world in usage of social media.

A survey from ~\cite{Needle:2023} shows that the optimal posting window would be between 9 am and noon, contradicting the earlier mentioned studies.

Ultimately, the studies reviewed show that figuring out the best time to post on Twitter is complicated and depends on different factors. The results varying so much, suggests that further research is needed.

\section{Data}


\paragraph{Research setup}
In our research specifically, we will use only tweets posted from the CET time zone, and for our results we will also consider only the optimal posting time for this time zone. This way, we avoid getting unwanted data as much as possible to get the most accurate results.

We will be using the amount of retweets (a function within Twitter that allows a user to share someone else's Tweet with their followers) as a measure of engagement for Twitter posts.  By considering the number of retweets, we try to capture the engagement and reach in the best possible way. In this case, the amount of retweets is our \textbf{dependent variable}. The time when the tweet is posted is our \textbf{independent variable}.

However, it's important to recognize a possible limitation when using retweet counts as a measure of engagement. Relying only on retweets may not represent completely how users engage with Tweets. Other features such as likes and comment also reflect a tweet's engagement.

\paragraph{Method}
To collect data for this research, we utilize the Twitter API, a tool that lets us access information from Twitter. First, we need permission, which comes in the form of credentials. These credentials include a consumer key, consumer secret, access token, and access token secret. With these credentials we can make API calls, to get data from Twitter.

The API responds by giving us a collection of tweets based on our request. Each tweet in our dataset has valuable details: a 'tweet id', when it was posted, the timezone of the user who posted the Tweet, and how many times it was retweeted. We will use a dataset of 20000 random tweets, to get the most accurate results.

\paragraph{Pre-processing} In the pre-processing phase of the data, we implement two important steps. First of all, we filter out Tweets that are not posted from the CET time zone, using metadata from the Twitter API. 

Secondly, we categorize the tweet timestamps into hourly slots. Categorizing the time in this way makes it easier to calculate the average retweet count for each time slot.

\begin{table}[h]
  \centering
  \begin{tabular}{p{1.5cm}p{4cm}p{3cm}p{2cm}p{2cm}}
    \toprule
    Tweet ID & Timestamp (CET) & RT Count \\
    \midrule
    123456 & 2023-01-15 10-11am & 51 \\
    123457 & 2023-01-15 3-4pm & 39 \\
    123458 & 2023-01-16 7-8am & 22 \\
    123459 & 2023-01-17 5-6pm & 112 \\
    123460 & 2023-01-17 1-2pm & 89 \\
    123461 & 2023-01-18 11-12pm & 44 \\
    123452 & 2024-01-19 3-4am & 8 \\
    
    % Add more rows as needed
    \bottomrule
  \end{tabular}
  \caption{Twitter Dataset *example, not real data*}
  \label{tab:example}
\end{table}


\section{Predicted Results}
For the results, we calculate the average retweet count of each time slot, by adding the retweet count for each tweet in the dataset to a total "time slot-count", and divide this total count by the total number of tweets in the time slot.
The predicted results are represented in the table below. We predict the results based on the relevant literature discussed earlier.


\begin{table}[hbtp]\centering
\begin{tabular}{p{2cm}p{4cm}p{4cm}p{3cm}p{3cm}}
    \toprule
    Time slot (CET) & AVG RT Count \\
    \midrule
    4pm-5pm & 70 \\
8am-9am & 68 \\
12pm-1pm & 65 \\
9pm-10pm & 62 \\
7am-8am & 60 \\
10am-11am & 60 \\
10pm-11pm & 58 \\
... & ... \\
4am-5am & 47 \\
3am-4am & 44 \\

    \bottomrule
  \end{tabular}
\caption{Predicted results}
\label{tbl: results}
\end{table}


\section{Discussion}
Our predicted results provide useful insights into how tweet timing relates to retweet behaviour. Specifically, tweets posted between 4 pm and 5 pm receive the highest average retweet count at 70, implying that the highest Twitter engagement occurs after work hours. 

Morning hours (8 am-9 am) also show significant engagement with an average retweet count of 68, implying that also in the beginning of the day Twitter users are often active.

Tweets posted in the middle of the night, between 3am and 5am, seem to perform the worst in terms of the amount of retweets.

Note that the results don't show when and in what frequency users are engaging in the time slots. For example, if Tweets posted between 4pm and 5pm have the most average retweets, that doesn't mean most users are active in this time slot, since they can also retweet later that day. At which times users are most active, could be a research question for another study.

\section{Conclusion}
This study aimed at uncovering how the timing of posting on Twitter affects the Tweet's engagement. We sought to find best and worst time to post a tweet (with best meaning getting maximum engagement) for the CET time zone.

The best time to post a tweet turned out to be between 4pm and 5pm, one hour earlier than we hypothesized to be the prime posting slot. There are logical reasons for this result. After work, most people have the time to wind down, relax, and check their social media, and interact with tweets the most. As per the worst time to post, this is between 3am and 4am, as we did correctly hypothesize. The reason for this is simple. Most people are asleep during this period, thus there is the lowest social media traffic.

We can also conclude that timing is a very important factor for engagement. Between the the most retweeted time slot and the least retweeted time slot, we can identify an average retweeting difference of 59.09\%. This difference is pretty significant, and essential for content creators that try to reach as many users as possible.



%%END YOUR PART


\bibliographystyle{chicago}
\bibliography{mybib.bib}

\end{document}
